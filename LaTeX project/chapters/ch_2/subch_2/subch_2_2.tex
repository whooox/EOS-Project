\normalsize{In this section, the fundamental principles and governing equations and each heat transfer mode are explored. The TZM-reflector tile is a highly constrained mechanical part that of the \acrshort{HS}. This part will be exposed to plasma radiation as well as the \acrshort{ECRH} beam. The heat transfer in this system is complex and need a good understanding of the theory to model it as good as possible.}
\\
\normalsize{\indent The heat loads on the \acrshort{ECRH} reflector tile are specified further in the thesis but the theory first needs to be explained, at least what will be needed for this task.}
\\
\subsection{General problem of heat exchange}
\normalsize{Heat exchange happens all the time and everywhere in nature, from the sun's radiative power to the heat transfer on the surface of the skin. Historically, heat was considered as some sort of {\it flow} that would flow from one hot object to another colder object \cites{ahtt6e}. The idea of an invisible fluid flowing from a body to another called {\it Caloric} was first considered to explain this heat transport. While the caloric theory of heat exchange is acceptable to consider such a concept for explaining heat transport, there are more modern approaches to heat exchange that will be discussed later \cites{ahtt6e}. The general problem of heat transfer involves understanding how thermal energy is transported from one place to another.
Heat exchange can be seen in many different situations and takes place in every medium and different modes. Leaving a hot house during winter with a door open and having the hot air making its way out, hummingbirds using a counterflow heat exchanger in their feet to regulate their body temperatures \cites{UDVARDY_1983} or inadvertedly touching a hot pan are all examples of heat exchange between mediums and objects \cites{ahtt6e}. It respects some rules such as the flow direction, from hot to cold.}
\\
\break
\normalsize{\indent Not only in nature but also in industry, heat is heavily used for/or is a product of chemical processes. Steam boilers convert chemical energy into heat to generate steam and then generate elctricity, this is also true for nuclear power plants, using the heat generated by fission reactions to generate steam. Heat is generated in combustion engines and needs to be evacuated meanwhile in fridges, heat in pumped to decrease the temperature in a chamber. Those devices all use some sort of heat transfer in order to work properly. Heat exchange is also present throughout of fusion devices at mane different levels such as inside the plasma as well as the first wall components and the pumping system.}
\\
\break
\normalsize{\indent Thermodynamics is a theory about the dynamics and conversion of different energy forms heavily developed during the 19th century. It provides a very good framework in which it is possible to built a theory of heat exchange \cites{ahtt6e}. It is possible to derive the equation of heat transfer from the laws of thermodynamics. The first law of thermodynamics for a closed system, as taught in engineering programs, takes the following form:}
\\
%\begin{equation}
    
%    \label{eqn:1stLoThDyn}
%\end{equation}

\subsection{Heat conduction}
\subsection{Convective heat transfer}
\subsection{Thermal radiation}