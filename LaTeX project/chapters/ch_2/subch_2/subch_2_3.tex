\normalsize{In this section, the fundamental principles and governing equations and each heat transfer mode are explored. The TZM-reflector tile is a highly constrained mechanical part that of the \acrshort{HS}. This part will be exposed to plasma radiation as well as the \acrshort{ECRH} beam. The heat transfer in this system is complex and need a good understanding of the theory to model it as good as possible.}
\\
\normalsize{\indent The heat loads on the \acrshort{ECRH} reflector tile are specified further in the thesis but the theory first needs to be explained, at least what will be needed for this task.}
\subsection{General problem of heat exchange}
\normalsize{Heat exchange happens all the time and everywhere in nature, from the sun's radiative power to the heat transfer on the surface of the skin. Historically, heat was considered as some sort of {\it flow} that would flow from one hot object to another colder object \cites{ahtt6e}. The idea of an invisible fluid flowing from a body to another called {\it Caloric} was first considered to explain this heat transport. While the caloric theory of heat exchange is acceptable to consider such a concept for explaining heat transport, there are more modern approaches to heat exchange that will be discussed later \cites{ahtt6e}. The general problem of heat transfer involves understanding how thermal energy is transported from one place to another. The modern aproach to heat transfer is the {\it kinetic} theory. Heat is defined to be the average Velocity of the particules within an system. This approach helps to understanding what heat is physically \cites{cengel2004heat}. Heat exchange can be seen in many different situations and takes place in every medium and different modes. Leaving a hot house during winter with a door open and having the hot air making its way out, hummingbirds using a counterflow heat exchanger in their feet to regulate their body temperatures \cites{UDVARDY_1983} or inadvertedly touching a hot pan are all examples of heat exchange between mediums and objects \cites{ahtt6e}. It respects some rules such as the flow direction, from hot to cold.}
\\
\break
\normalsize{\indent Not only in nature but also in industry, heat is heavily used for/or is a product of chemical processes. Steam boilers convert chemical energy into heat to generate steam and then generate elctricity, this is also true for nuclear power plants, using the heat generated by fission reactions to generate steam. Heat is generated in combustion engines and needs to be evacuated meanwhile in fridges, heat in pumped to decrease the temperature in a chamber. Those devices all use some sort of heat transfer in order to work properly. Heat exchange is also present throughout of fusion devices at many different levels such as inside the plasma as well as the first wall components and the pumping system. Correctly modelling heat exchange as well as understanding the physical phenomena behind the transfer of heat in thermal machines.}
\\
\break
\normalsize{\indent Thermodynamics is a theory about the dynamics and conversion of different energy forms heavily developed during the 19th century. It provides a very good framework in which it is possible to built a theory of heat exchange \cites{ahtt6e}. It is possible to derive the equation of heat transfer from the laws of thermodynamics. The first law of thermodynamics for a closed system, as taught in engineering programs, takes the following form:}

\subsection{Heat conduction}
\normalsize{Conduction is the transfer of energy from the more energetic particles of a substance to the adjacent less energetic ones as a result of interactions between the particles \cites{cengel2004heat}. Conduction can take place in solids, liquids, or gases. In fluids, conduction is due to the collision and diffusion of the molecules during their random motion. In solids, conduction is due to the combination of vibration of the molecules in a lattice and the energy transport by free electrons.}
\\
\break
\normalsize{\indent Lets consider steaty-state heat conduction through a plane wall of thickness $L  = \Delta x$ and area $A$. The difference of temperature or {\it gradient} is measured and is written $\Delta T = T_2 - T_1$. Experience has shown that the rate of heat transfer $\dot{Q}_{cond}$ through the wall would double when the temperature gradient across the wall {\it or} when when the area $A$ normal to the direction of heat transfer doubles. The rate of heat transfer would be halved when the thickness $L$ doubled. Qualitavively, it is possible to conclude that the rate of heat conduction through a plane wall is proportional to the temperature gradient across the layers and the heat transfer area, but is inverssely proportional to the thickness of the layer. The relation between the quantities is:}
\\
\begin{equation}
    (Rate \ of \ heat \ conduction) \propto \frac{(Area)(Temperature \ gradient)}{Thickness}
    % \tagaddtext{[\si{\watt}]}
    \label{eqn:CondEq}
\end{equation}
\\
\normalsize{\indent Analytically, the mathematical law describing the conduction of heat is {\bfseries Fourier's law of heat conduction}. The coefficient $k$ is the thermal conductivitiy, which is the abilitiy of a certain material to conduct heat. It is possible to write this law using quantities and the equation is :}
\\
\begin{equation}
    \dot{Q}_{cond} = -kA\frac{T_2 - T_1}{\Delta x} = -kA\frac{\Delta T}{\Delta x}
    \tagaddtext{[\si{\watt}]}
    % \tagaddtext{[\si{\watt}]}
    \label{eqn:CondEqInt}
\end{equation}
\\
\normalsize{When $\Delta x \rightarrow 0$,  the one-dimensionnal differential form of the equation is written:}
\\
\begin{equation}
    \dot{Q}_{cond} = -kA\frac{dT}{dx}
    \tagaddtext{[\si{\watt}]}
    \label{eqn:CondEq}
\end{equation}
\\
\normalsize{\indent On the left hand side of the heat conduction equation \refeq{eqn:CondEq} $\dot{Q}_{cond}$ decribes the time derivative or temporal rate of change of the heat flux flowing though a surface. On the the right hand side of the equation, $k$ is the thermal conductivity. Usually, the thermal conductivity is a function of the temperature itself making this differential equation nonlinear. }
\\
\subsection{Convective heat transfer}
\normalsize{It was mentionned earlier that there are three basic mechanisms of heat transfer: conduction, convection, and radiation. Conduction and convection are similar in that both mechanisms require the presence of a material medium. But they are different in that convection requires the presence of fluid motion. Heat transfer through a solid is always by conduction, since the molecules of a solid remain at relatively fixed positions. Heat transfer through a liquid or
gas, however, can be by conduction or convection, depending on the presence of any bulk fluid motion. Heat transfer through a fluid is by convection in the presence of bulk fluid motion and by conduction in the absence of it.}
\subsection{Thermal radiation}