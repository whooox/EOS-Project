\section{PLASMA HEAT LOAD}
\normalsize{The main heat load on the \acrshort{ECRH} reflector tile is the plasma heat load. This reflector tile is directly exposed to the plasma. The magnitude of the thermal load depends greatly on the position of the components relative to the plasma. Components in direct contact with the plasma boundary, such as the divertor targets are loaded primarily through convected power loads, whilst the other components such as the wall panels are primarily loaded by radiation from the plasma. For the design of the \acrshort{PFCs} it is important to know for the different plasma scenarios how the heating is split between the different components and how it is divided between convected and radiated fractions. In addition to the plasma heat load, the ECRH beam will also contribute to a certain amount to the heating of the tile assembly. Because this assembly going to be under high functionnal heat loads, it is important to correctly model the heat load of the plasma on the ECRH reflector tile.}
\\
\break
\normalsize{\indent Stationary operation, envisaged at \acrshort{W7-X}, demands an actively cooled wall protection system. The design of the approximate 115 m2 wall protection depends mainly on the distance of the last closed flux surface to the plasma facing components. The area of the outboard part, where expected heat loads do not exceed 100 – 200 kW/m2 will be mainly covered with actively cooled stainless steel panels (approximately 70 m2). For the rest, in particular on the inboard side, where the plasma-wall distance is smaller, higher heat loads are expected (up to 0.5 MW/m2). In this area (about 45 m2) graphite tiles clamped on watercooled CuCrZr structures, will protect components behind the heat shield.}
\\
\break
\normalsize{\indent To respect \acrshort{OP2} operational and to stay consistant with the previous calculations,}