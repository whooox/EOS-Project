\chapter{CONCLUSION AND RECOMMENDATIONS}
\normalsize{The aim of this thesis was to investigate the thermal and mechanical behavior of the \acrshort{ECRH} reflector tile of \acrshort{W7-X} and to validate the proper functioning of the assembly for future operation phase \acrshort{OP2}. Through a comprehensive modelling of the tile assembly and the diverse use of finite element analysis, this study has provided significant insights into the thermal performances of the tile assembly.}
\\
\break
\normalsize{\indent Our findings reveal that the new tile design proposed by Victor Bykov does indeed help reduce power flow through the tile reducing the temperature (i.e. \acrshort{CuCrZr} heat sink has a decrease of about 3,33\% of initial design temperature). It was also found out that the film coefficient used by Jiawu Zhu in his 2019 study is not likely to be attained during operation. This was found to lead to overheating of the heat sink which could trigger a recrystallization of the bronze alloy and negatively impact the structural integrity of the tile assembly. Further discussion with Axel Lorenz stated that this shouldn’t be an issue with the alloy currently used but should still be carefully assessed. To assess the duration at which the heat sink over heats, a transient thermal analysis was performed and the results showed overheating about 143s after pulse start and steady-state reached after about 250s. These results have important implications for tile assembly since its behavior limits the plasma pulse duration, suggesting that an overheating subsystem could impact the whole performance of the machine.}
\\
\break
\normalsize{\indent However, it is important to acknowledge the limitations of this study. It is not possible to accurately model the system when considering 100\% thermal contact. These limitations may have influenced the calculated temperature fields, and addressing them further in future research could yield even more robust conclusions. A first approach was to run multiphysics simulations coupling thermal and structural calculations. This was a good was to capture complex coupled phenomena but the bolting system could not be properly implemented because of instabilities during solving.}
\\
\break
\normalsize{\indent In light of our findings, several aven ues for future research have been identified. It is recommended to continue the development of multiphysical models of the tile assembly to further analyze the in-depth mechanics of its functioning. It is also proposed, in a more model/experiment approach, to consider the installation of temperature sensor on different parts of the tile assembly to gather information and compare the numerical model to real data. This could help steer the development of further more advanced models of this critical part of the \acrshort{PFCs}. These directions hold promise for further advancing our understanding of the thermal and mechanical behavior of \acrshort{PFCs} and addressing the questions that remain unanswered.}
\\
\break
\normalsize{\indent In conclusion, this thesis has contributed to the field of nuclear fusion by providing good insight on heavily constrained plasma facing components, which remain an engineering challenge to reach commercial fusion. The insights gained from this research offer valuable perspectives on the thermal and mechanical behavior of the \acrshort{ECRH} reflector tile of \acrshort{W7-X} and pave the way for future investigations that can build upon this foundation.}
\section{Recommendations}
\normalsize{After the analysis if the different results, it is recommended to pursue the development of the multiphysical model of the tile assembly and the implementation of the bolting system in that model. This means the solving of the convergence issues (mostly due to faulty contact configuration) seen during the development of the early coupled fields analyses. While computationally expensive and complex modelling, such a multiphysical model could be of great help understanding the real life behavior of the tile assembly and move from a rudimentary model with 100 \% thermal contact to a better description of real life phenomena.}
\\
\break
\normalsize{\indent The installation of a temperature sensor is also recommended to collect data and refine the numerical model based on the collected data. This also allows to validate the numerical model and monitor the evolution of the temperature and subsequently the state of the assembly during the operational phase.} 